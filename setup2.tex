
\usepackage[width=15.50cm, height=24.00cm]{geometry}
\usepackage[english]{babel}
\usepackage[utf8]{inputenc}
\usepackage[T1]{fontenc}
\usepackage{dsfont}
\usepackage{amsmath}
\usepackage{amsfonts}
\usepackage{amssymb}
\usepackage{graphicx}
\usepackage{paracol}
\usepackage{xparse}
\usepackage{sidecap}
\usepackage[makeroom]{cancel}
\usepackage{capt-of}
\usepackage{caption}
\usepackage[dvipsnames]{xcolor}
\usepackage{xpatch}
\usepackage{subcaption}
\usepackage[most]{tcolorbox}
\usepackage{lipsum}
\usepackage{float}
\usepackage{imakeidx}
\usepackage{wrapfig}
\usepackage{marginnote}
\usepackage{ upgreek }
\usepackage{bm}
\usepackage{enumerate}
\usepackage{mathrsfs} 
\usepackage{tikzit}
\usepackage{multirow}

\usetikzlibrary{arrows.meta}
\input{style.tikzstyles}
\graphicspath{{Images/}}


\makeindex[columns=3, title=Indice Analitico, intoc]

\captionsetup{font = {it, small}, labelfont={color=NavyBlue, bf}}

\newcommand{\mat}[1]{\mathbf{#1}}
\newcommand{\vet}[1]{\pmb{#1}}
\newcommand{\vett}[1]{\boldsymbol{#1}}
\newcommand{\A}{\mat{A}}
\newcommand{\B}{\mat{B}}
\newcommand{\C}{\mat{C}}
\newcommand{\D}{\mat{D}}
\newcommand{\G}{\mat{G}}
\newcommand{\R}{\mat{R}}
\newcommand{\Q}{\mat{Q}}
\newcommand{\K}{\mat{K}}
\newcommand{\V}{\mat{V}}
\renewcommand{\P}{\mat{P}}
\renewcommand{\O}{\mat{O}}
\renewcommand{\S}{\mat{S}}
\newcommand{\T}{\mat{T}}
\newcommand{\I}{\mat{I}}
\newcommand{\x}{\vet{x}}
\renewcommand{\u}{\vet{u}}
\newcommand{\y}{\vet{y}}
\newcommand{\z}{\vet{z}}
\newcommand{\dx}{\dot{\x}}
\newcommand{\xp}{\x^+}
\newcommand{\jordan}{\vet J}

\newcommand{\reach}{\mathcal R}
\newcommand{\ctr}{\mathcal C}
\newcommand{\rgram}{\mat W_\reach}
\newcommand{\cgram}{\mat W_\ctr}
\newcommand{\image}[1]{\textrm{Im}\left\{#1\right\}}
\newcommand{\kernel}[1]{\textrm{Ker}\left\{#1\right\}}
\newcommand{\dimension}[1]{\textrm{dim}\{#1\}}
\newcommand{\rank}[1]{\textrm{rank}\{#1\}}
\newcommand{\unobs}{\mathcal{N_O}}
\newcommand{\uncostr}{\mathcal{N_C}}
\newcommand{\ogram}{\mat W_{\mathcal O}}
\newcommand{\construcgram}{\mat W_{\mathcal C n}}
\newcommand{\jset}{\mathcal D}
\newcommand{\fset}{\mathcal C}
\newcommand{\hsol}{\mathds E}

\renewcommand{\Re}[1]{\mathds R\textrm{e}\{#1\}}

\renewcommand{\L}{\mathscr L}
\newcommand{\Z}{\mathscr Z}
\newcommand{\stm}{\mat{\Phi}}

\renewcommand{\vector}[1]{\begin{pmatrix} #1 \end{pmatrix}}
\renewcommand{\matrix}[1]{\begin{bmatrix} #1 \end{bmatrix}}

\newcommand{\de}[1]{\textbf{\textcolor{NavyBlue}{#1}}}
\newcommand{\bfcolor}[1]{\renewcommand*{\textbf}[1]{{\bfseries {\color{#1}##1}}}}


\newenvironment{note}{
	\bfcolor{CadetBlue}
	\textbf{Note: }
}{ }
\tcolorboxenvironment{note}{
	boxrule=0pt,
	boxsep=0pt,
	colback={White!100!CadetBlue},
	enhanced jigsaw, 
	borderline west={2pt}{0pt}{CadetBlue},
	sharp corners,
	before skip=5pt,
	after skip=10pt,
	breakable,
}


\newcounter{numth}
\numberwithin{numth}{chapter}
\newenvironment{theorem}{
	\noindent
	\refstepcounter{numth}
	
	%	\renewcommand{\de}[1]{ { \color{ForestGreen} \textbf{#1} } }
	
	{\color{NavyBlue}\textbf{Theorem \thenumth:}}
}{
}
\tcolorboxenvironment{theorem}{
	boxrule=0pt,
	boxsep=0pt,
	colback={White!90!NavyBlue},
	enhanced jigsaw, 
	borderline west={2pt}{0pt}{NavyBlue},
	sharp corners,
	before skip=5pt,
	after skip=5pt,
	breakable,
}

\newcounter{numproof}
\numberwithin{numproof}{chapter}
\newenvironment{proof}{
	\noindent
	\refstepcounter{numproof}
	
	%	\renewcommand{\de}[1]{ { \color{ForestGreen} \textbf{#1} } }
	
	{\color{NavyBlue}\textbf{Proof \thenumproof:}}
}{
}
\tcolorboxenvironment{proof}{
	boxrule=0pt,
	boxsep=0pt,
	colback={White!100!NavyBlue},
	enhanced jigsaw, 
	borderline west={2pt}{0pt}{NavyBlue},
	sharp corners,
	before skip=5pt,
	after skip=5pt,
	breakable,
}


\newcounter{numdem}
\numberwithin{numdem}{chapter}
\newenvironment{demonstration}{
	\noindent
	\refstepcounter{numdem}
	
	%	\renewcommand{\de}[1]{ { \color{ForestGreen} \textbf{#1} } }
	
	{\color{ForestGreen}\textbf{Demonstration \thenumdem:}}
}{
}
\tcolorboxenvironment{demonstration}{
	boxrule=0pt,
	boxsep=0pt,
	colback={White!90!ForestGreen},
	enhanced jigsaw, 
	borderline west={2pt}{0pt}{ForestGreen},
	sharp corners,
	before skip=5pt,
	after skip=5pt,
	breakable,
}

\newcounter{exercises}
\numberwithin{exercises}{chapter}
\newenvironment{exercise}[1]{
	\bfcolor{Periwinkle}
	\noindent
	\refstepcounter{exercises}
	{\color{Periwinkle}\textbf{Exercise \theexercises#1}  } 
	\vspace{3mm}
	
	\noindent
}{
}


\tcolorboxenvironment{exercise}{
	boxrule=0pt,
	boxsep=0pt,
	colback={White!90!Periwinkle},
	enhanced jigsaw, 
	borderline west={2pt}{0pt}{Periwinkle},
	sharp corners,
	before skip=10pt,
	after skip=10pt,
	breakable,
}

\newcounter{examples}
\numberwithin{examples}{chapter}
\newenvironment{example}[1]{
	\bfcolor{Periwinkle}
	\noindent
	\refstepcounter{examples}
	{\color{Periwinkle}\textbf{Example \theexamples#1}  } 
	\vspace{3mm}
	
	\noindent
}{
}


\tcolorboxenvironment{example}{
	boxrule=0pt,
	boxsep=0pt,
	colback={White!90!Periwinkle},
	enhanced jigsaw, 
	borderline west={2pt}{0pt}{Periwinkle},
	sharp corners,
	before skip=10pt,
	after skip=10pt,
	breakable,
}
